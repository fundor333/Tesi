% !TEX encoding = UTF-8
% !TEX TS-program = pdflatex
% !TEX root = ../Tesi.tex
% !TEX spellcheck = it-IT

%************************************************
\chapter{Realizzazione}
\label{cap:realizzazione}

Il progetto è stato realizzato attraverso la divisione in subproblemi. Questo permette la sostituzione di una componente con un'altra in modo rapido ed sicuro in quanto viene a essere modificata solo la parte del codice interessata dall'operazione e non tutte le altre componenti, che rimangono invariate. 

Questo ha portato anche alla ricerca di alcune librerie per eseguire le operazioni necessarie alle funzioni scelte.

\section{Divisione in sottoproblemi}
Per semplificare lo sviluppo di questa applicazione e il suo sviluppo è stato diviso il progetto in quattro componenti indipendenti che dialogano tra di loro, in modo da rendere più isolate le varie componenti, rendendole meno soggette a errori non previsti. 
Queste si dividono in:
\begin{itemize}
\item Interfaccia grafica
\item Lettura dei file in ingresso
\item Gestione ed elaborazione dei dati
\item Scrittura dei file in uscita
\end{itemize}

Questa strutturazione permette anche il riciclo del codice utilizzato per risolvere i sottoproblemi in altri progetti con la stessa necessità o la facile estensione del plugin in un secondo momento

\subsection{Interfaccia grafica}
Al momento della progettazione dell'applicazione è stato necessario pensare come realizzare l'inserimento dei dati da elaborare da parte dell'utente. Questo, solitamente, viene fatto in due modi:
\begin{itemize}

\item Attraverso l'uso di un \textbf{terminale} in cui vengono passati dei comandi che eseguono le operazioni richieste e mostra a schermo tutti i dati dei comandi eseguiti. Questo però obbliga l'utente a sapere i comandi necessari alla esecuzione dei task e ritorna a schermo dati di poco interesse per l'utente medio, molto utile invece se sei un programmatore e vuoi controllare come procedono i vari passi dei comandi eseguiti e permette di vedere subito ove avvengono gli errori e cosa li generano. 

\item Attraverso l'uso di una \textbf{interfaccia grafica o GUI} in cui viene selezionato, attraverso l'uso di bottoni, menu e interuttori le funzioni che si vogliono eseguire e si danno i file in input con un selettore grafico. Permette all'utente di non interessarsi a come è stato realizzata l'operazione richiesta, di non cunsultare guide e manuali con gli elenchi dei comandi disponibili e di ricevere a schermo solo le informazioni di cui è realmente interessato. Oltre a essere la scelta più pratica per un nuovo utente è anche quella che da al programmatore che sviluppa questa applicazione la maggior libertà e indipendenza. Infatti la presenza di una grafica comune tra aggiornamenti diversi permette al programmatore di riscrivere interamente i comandi che vengono eseguiti al di sotto dell'interfaccia senza dover avvisare l'utente del cambiamento dei comandi all'interno del terminale.

\end{itemize}

Per come è strutturato QGis è possibile avere entrambi implementati nel plugin ma è stato preferito dare maggiore attenzione alla componente grafica anche se è comunque possibile aprire il terminale di QGis ed eseguire dei comandi. 

In particolare in questa applicazione è stata implementata l'interfaccia grafica in quanto risulta molto più adatta all'uso effettivo dell'applicazione. 

% TODO inserire qui immagine della gui

Infatti come si vede nell'immagine [link immagine] per il progetto in esame è necessario soltanto una gui di inserimento dati in cui, in base al tipo di file inserito e al tipo di output,li elabora correttamente. Questo, su terminale, risulta invece più difficile in quanto bisognerebbe avere una lista dei driver necessari per la corretta lettura e inserirli assieme al file da leggere\footnote{Questo viene fatto automaticamente dall'interfaccia grafica selezionando il file desiderato}.

\subsubsection{Qt4}

Le librerie grafiche per Python\footnote{Il linguaggio in cui è stato scritto il plugin} sono molte e varie ma la scelta è stata abbastanza facile: QT.

Questa scelta è stata dettata dal desiderio di ridurre al minimo le dipendenze esterne dal plugin. Questo comporta l'utilizzo preferenziale delle librerie utilizzate da QGis per semplificare l'installazione delle dipendenze. In particolare, con QGis è stato necessario scegliere anche la versione di QT con cui sviluppare il plugin. La scelta è stata fatta su QT4 ovvero la funzione supportata dalla versione 2.0 in poi di QGis.  Questo permette un enorme supporto di versioni di QGis da parte del plugin. 

La versione QT5 che verrà impiegata per le future versioni di QGis è già interamente supportata dall'applicazione. Questo è stato possibile attraverso un attento sviluppo attraverso funzioni non deprecate e controllo continuo delle nuove funzionalità di QT5 e delle modifiche che questa fa nel funzionamento complessivo in modo di essere già compatibile quando sarà necessario passare a QT5 in quanto non sarà più prerequisito di QGis.

\subsection{Gestione input}
\subsection{Gestione dati}
\subsubsection{Grass}
\subsubsection{Osgeo}
\subsection{Gestione output}
\section{Librerie scelte}

