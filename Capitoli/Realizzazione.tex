% !TEX encoding = UTF-8
% !TEX TS-program = pdflatex
% !TEX root = ../Tesi.tex
% !TEX spellcheck = it-IT

%************************************************
\chapter{Realizzazione}
\label{cap:realizzazione}

Il progetto è stato realizzato attraverso la divisione in subproblemi. Questo permette la sostituzione di una componente con un'altra in modo rapido ed sicuro in quanto viene a essere modificata solo la parte del codice interessata dall'operazione e non tutte le altre componenti, che rimangono invariate. 

Questo ha portato anche alla ricerca di alcune librerie per eseguire le operazioni necessarie alle funzioni scelte.

\section{Divisione in sottoproblemi}
Per semplificare lo sviluppo di questa applicazione e il suo sviluppo è stato diviso il progetto in quattro componenti indipendenti che dialogano tra di loro, in modo da rendere più isolate le varie componenti, rendendole meno soggette a errori non previsti. 
Queste si dividono in:
\begin{itemize}
\item Interfaccia grafica
\item Lettura dei file in ingresso
\item Gestione ed elaborazione dei dati
\item Scrittura dei file in uscita
\end{itemize}

Questa strutturazione permette anche il riciclo del codice utilizzato per risolvere i sottoproblemi in altri progetti con la stessa necessità o la facile estensione del plugin in un secondo momento

\subsection{Interfaccia grafica}
Al momento della progettazione dell'applicazione è stato necessario pensare come realizzare l'inserimento dei dati da elaborare da parte dell'utente. Questo, solitamente, viene fatto in due modi:
\begin{itemize}

\item Attraverso l'uso di un \textbf{terminale} in cui vengono passati dei comandi che eseguono le operazioni richieste e mostra a schermo tutti i dati dei comandi eseguiti. Questo però obbliga l'utente a sapere i comandi necessari alla esecuzione dei task e ritorna a schermo dati di poco interesse per l'utente medio, molto utile invece se sei un programmatore e vuoi controllare come procedono i vari passi dei comandi eseguiti e permette di vedere subito ove avvengono gli errori e cosa li generano. 

\item Attraverso l'uso di una \textbf{interfaccia grafica o GUI} in cui viene selezionato, attraverso l'uso di bottoni, menu e interuttori le funzioni che si vogliono eseguire e si danno i file in input con un selettore grafico. Permette all'utente di non interessarsi a come è stato realizzata l'operazione richiesta, di non cunsultare guide e manuali con gli elenchi dei comandi disponibili e di ricevere a schermo solo le informazioni di cui è realmente interessato. Oltre a essere la scelta più pratica per un nuovo utente è anche quella che da al programmatore che sviluppa questa applicazione la maggior libertà e indipendenza. Infatti la presenza di una grafica comune tra aggiornamenti diversi permette al programmatore di riscrivere interamente i comandi che vengono eseguiti al di sotto dell'interfaccia senza dover avvisare l'utente del cambiamento dei comandi all'interno del terminale.

\end{itemize}



\subsubsection{Qt4}
\subsection{Gestione input}
\subsection{Gestione dati}
\subsubsection{Grass}
\subsubsection{Osgeo}
\subsection{Gestione output}
\section{Librerie scelte}

