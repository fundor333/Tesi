% !TEX encoding = UTF-8
% !TEX TS-program = pdflatex
% !TEX root = ../Tesi.tex
% !TEX spellcheck = it-IT

%************************************************
\chapter{Realizzazione}
\label{cap:realizzazione}

Il progetto è stato realizzato attraverso la divisione in subproblemi. Questo permette la sostituzione di una componente con un'altra in modo rapido ed sicuro in quanto viene a essere modificata solo la parte del codice interessata dall'operazione e non tutte le altre componenti, che rimangono invariate.

Questo ha portato anche alla ricerca di alcune librerie per eseguire le operazioni necessarie alle funzioni scelte.

\section{Divisione in sottoproblemi}
Per semplificare lo sviluppo di questa applicazione e il suo sviluppo è stato diviso il progetto in quattro componenti indipendenti che dialogano tra di loro, in modo da rendere più isolate le varie componenti, rendendole meno soggette a errori non previsti.
Queste si dividono in:
\begin{itemize}
\item Interfaccia grafica
\item Lettura dei file in ingresso
\item Gestione ed elaborazione dei dati
\item Scrittura dei file in uscita
\end{itemize}

Questa strutturazione permette anche il riciclo del codice utilizzato per risolvere i sottoproblemi in altri progetti con la stessa necessità o la facile estensione del plugin in un secondo momento

\subsection{Interfaccia grafica}
Al momento della progettazione dell'applicazione è stato necessario pensare come realizzare l'inserimento dei dati da elaborare da parte dell'utente. Questo, solitamente, viene fatto in due modi:
\begin{itemize}

\item Attraverso l'uso di un \textbf{terminale} in cui vengono passati dei comandi che eseguono le operazioni richieste e mostra a schermo tutti i dati dei comandi eseguiti. Questo però obbliga l'utente a sapere i comandi necessari alla esecuzione dei task e ritorna a schermo dati di poco interesse per l'utente medio, molto utile invece se sei un programmatore e vuoi controllare come procedono i vari passi dei comandi eseguiti e permette di vedere subito ove avvengono gli errori e cosa li generano.

\item Attraverso l'uso di una \textbf{interfaccia grafica o GUI} in cui viene selezionato, attraverso l'uso di bottoni, menu e interuttori le funzioni che si vogliono eseguire e si danno i file in input con un selettore grafico. Permette all'utente di non interessarsi a come è stato realizzata l'operazione richiesta, di non cunsultare guide e manuali con gli elenchi dei comandi disponibili e di ricevere a schermo solo le informazioni di cui è realmente interessato. Oltre a essere la scelta più pratica per un nuovo utente è anche quella che da al programmatore che sviluppa questa applicazione la maggior libertà e indipendenza. Infatti la presenza di una grafica comune tra aggiornamenti diversi permette al programmatore di riscrivere interamente i comandi che vengono eseguiti al di sotto dell'interfaccia senza dover avvisare l'utente del cambiamento dei comandi all'interno del terminale.

\end{itemize}

Per come è strutturato QGis è possibile avere entrambi implementati nel plugin ma è stato preferito dare maggiore attenzione alla componente grafica anche se è comunque possibile aprire il terminale di QGis ed eseguire dei comandi.

In particolare in questa applicazione è stata implementata l'interfaccia grafica in quanto risulta molto più adatta all'uso effettivo dell'applicazione.

%TODO inserire qui immagine della gui

Infatti come si vede nell'immagine [link immagine] per il progetto in esame è necessario soltanto una gui di inserimento dati in cui, in base al tipo di file inserito e al tipo di output,li elabora correttamente. Questo, su terminale, risulta invece più difficile in quanto bisognerebbe avere una lista dei driver necessari per la corretta lettura e inserirli assieme al file da leggere\footnote{Questo viene fatto automaticamente dall'interfaccia grafica selezionando il file desiderato}.

\subsubsection{Qt4}

Le librerie grafiche per Python\footnote{Il linguaggio in cui è stato scritto il plugin} sono molte e varie ma la scelta è stata abbastanza facile: QT.

Questa scelta è stata dettata dal desiderio di ridurre al minimo le dipendenze esterne dal plugin. Questo comporta l'utilizzo preferenziale delle librerie utilizzate da QGis per semplificare l'installazione delle dipendenze. In particolare, con QGis è stato necessario scegliere anche la versione di QT con cui sviluppare il plugin. La scelta è stata fatta su QT4 ovvero la funzione supportata dalla versione 2.0 in poi di QGis.  Questo permette un enorme supporto di versioni di QGis da parte del plugin.

La versione QT5 che verrà impiegata per le future versioni di QGis è già interamente supportata dall'applicazione. Questo è stato possibile attraverso un attento sviluppo attraverso funzioni non deprecate e controllo continuo delle nuove funzionalità di QT5 e delle modifiche che questa fa nel funzionamento complessivo in modo di essere già compatibile quando sarà necessario passare a QT5 in quanto QT4 non sarà più prerequisito di QGis.

Ovviamente si poteva avere lo stesso risultato utilizzando altre librerie grafiche esterne o quelle di pyhton ma è stato scelto QT anche per avere una coerenza grafica tra le varie finestre che vanno ad aprirsi. Infatti se venisse utilizzata un altra libreria grafica sarebbe necessario una rielaborazione della grafica standard della libreria in modo che si comporti coerentemente rispetto a QT per ogni sistema operativo, il che porterebbe a una complicazione del codice non necessaria.

\subsection{Leggere e scrivere i dati}
Il plugin prende in ingresso i file contenenti i dati necessari per l'applicazione al modello matematico descritto dall'equazione \ref{eq:main}. Quindi, una volta eseguite le operazioni produce l'output corrispondente all'erosione per il numero di anni dati in input.

Le operazioni di lettura e scrittura sono le operazioni più sensibili in questo applictivo in quanto non c'è modo di controllare la correttezza dei dati in input dal punto di vista fisico/geologico, solo dal punto di vista informatico di buona formattazione.

\subsubsection{Input}
Per una questione di coerenza rispetto a QGis è stato deciso di passare i dati in input come file, in modo che possano essere facilmente condivisi e possano interagire più facilmente con QGis stesso. Per fare ciò è stato necessario fare una ricerca sui tipi di file utilizzati per i dati georefernziati.

Per questo motivo è stata fatta una ricerca sui formati di file supportati nativamente da QGis che fossero di tipo raster\footnote{Unico tipo di dati che il plugin accetta in input in quanto è l'unica struttura dati adatta a contenere i dati che ci servono per l'elaborazione} e quali fossero supportati attraverso plugin esterni.

E' stata quindi usata una delle funzioni del core di QGis per il riconoscimento automatico dei formati riconosciuti da QGis in modo da eseguire l'elaborazione del singolo file col driver corrispondente installato migliore.

Questo permette di avere un plugin più elastico, in quanto non necessita di aggiornamento se viene aggiunto il supporto a un tipo aggiuntivo di file per QGis ma recupera direttamente dalle configurazioni di QGis i dati necessari per la lettura, decodifica e rielaborazione del file in questione.

In oltre, il codice di gestione per i file in input è integrato nell'interfaccia grafica in modo da rendere semplice l'inserimento di dati all'interno del programma stesso. Supporta comunque l'utilizzo attraverso il terminale di QGis anche se ne è sconsigliato l'uso ai meno esperti e a chi si approccia per la prima volta a questo plugin.
%TODO inserire immagine apertura file

Come vedete dall'immagine il plugin apre una finestra di sistema differente in base al sistema operativo utilizzato, in quanto la gestione della scelta del file è delagato a esso. Una volta avvenuta la scelta questa finestra si chiuderà passando la stringa corrispondente al path del file aperto al campo adiacente presente nella gui principale. %TODO inserire riferimento a immagine gui generale
Se l'utente lo desidera puo' editare o direttamente inserire il path al file scelto direttamente nel campo senza aprire la finestra per la scelta della cartella.

%TODO Spiegare la lettura dei raster e il loro salvataggio dentro enuple

\subsubsection{Output}
Come per l'input, l'output può essere gestito in due modi: attraverso il terminale python o via interfaccia grafica.

In entrambi i casi è possibile indicare il path e il nome del file di output e il formato in cui lo si vuole salvare.
Il formato è scelto dall'utente da un elenco di formati disponibili nativamente da QGis con l'aggiunta dei formati disponibili da i plugin installati\footnote{Solo se questi registrano il formato nell'elenco nativamente presente in QGis come indicato dalle API per sviluppatori.} con driver per la scrittura. Per come è costruito QGis è possibile che alcuni formati non siano disponibili per la scrittura in quanto i driver sono di solo lettura. In questo caso, indipendentemente dalla scelta di utilizzare il terminale o una gui il plugin lancerà una finestra con l'errore corrispondente all'errore lanciato dal driver. Questo viene fatto in quanto è possibile che i formati in input non possono essere correttamente codificati nel formato di output scelto.

In ogni caso si consiglia che il formato di input e output dei file sia lo stesso per avere il minor problemi dati dalla decodifica e codifica degli stessi.

Nel casos i driver non vengono registrati all'interno dell'elenco dei driver per l'output di QGis seguendo le api sarà necessario implementarle sequendole per poterle utilizzare all'interno del plugin.

%TODO Spiegare la scrittura di raster
\subsection{Elaborazione dati}
Una volta ricevuti i dati in input si pensa a elaborarli. Per farlo è stato necessario generare una struttura dati idonea ai formati in input. E' stato quindi deciso di implementare l'elaborazione attraverso la struttura dati raster. A sua volta i dati geomorfici del raster sono all'interno di una matrice, che è l'oggetto con cui il plugin esegue le operazioni.

In oltre è stato necessario trovare una o più librerie che implementassero le funzionalità richieste per l'elaborazione dei dati raster mantenendo al minimo le dipendenze esterne. In caso di assenza di librerie si sarebbe dovuto implementare tutto da zero. Per nostra fortuna sono disponibili alcune librerie che eseguono calcoli in raster.

\subsubsection{Grass}
Plugin standard che da accesso a tutte le funzionalità avanzate di GRASS GIS. Dal punto di vista tecnico non è una libreria ma un insieme di librerie che gestiscono raster e elaborazioni avanzate dei dati. Essendo di facile installazione e, in alcune versioni, distribuito direttamente con QGis è un ottimo sistema per avere facilmente una libreria per l'elaborazione dei raster.

Il plugin descritto in questa tesi richiede l'istallazione di Grass per il suo corretto funzionamento anche se non usa direttamente la libreria di Grass. 

La libreria di Grass non è stata usata direttamente ma solo una delle sue componenti per avere maggiore libertà nei formati disponibili di file. Infatti, a meno di un ridotto numero di plugin, non è possibile aumentare i tipi di file supportati da Grass in quanto si limitano solo ai formati forniti con l'installazione standard di QGis.

\subsubsection{Osgeo}
Una delle librerie utilizzate da Grass. Formalmente è una libreria per il supporto di operazioni su raster, in pratica è un porting di Numpy per i formati di file contenenti dati georeferenziati.

Nel nostro caso è stato utilizzato per elaborare i dati dei file raster.