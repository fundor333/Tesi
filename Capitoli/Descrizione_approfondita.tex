% !TEX encoding = UTF-8
% !TEX TS-program = pdflatex
% !TEX root = ../Tesi.tex
% !TEX spellcheck = it-IT

%************************************************
\chapter{Descrizione dell'argomento}
\label{cap:descrizione}
%************************************************
\section{Problema da risolvere} 
Lo scopo di questa tesi è quello di calcolare e visualizzare in modo corretto l'erosione del territorio nell'area geografica scelta. Come già accennato nell'introduzione, si puo' scegliere qualunque area georgrafica superiore a un metro quadro\footnote{Dimensione minima sotto la quale i file formato raster non possono andare}.

Il problema puo' essere riassunto nell'equazione 
\begin{equation} \label{eq:main}
A = R * K * LS * C * P
\end{equation}

in cui
\begin{itemize}
	\item \textbf{A} corrispondente alla perdita di suolo annua
	\item \textbf{R} erosività delle precipitazioni
	\item \textbf{K} erodibilità del suolo
	\item \textbf{LS} rapporto tra la lunghezza e la pendenza
	\item \textbf{C} fattore di copertura del suolo
	\item \textbf{P} i parametri delle misure di prevenzione dell'erosione espressi sotto forma di matrice
\end{itemize}

Tutti questi dati sono forniti sotto forma di file che vengono poi elaborati per ottenere il raster corrispondente alla perdita di suolo annuo. Fa eccezzione il parametro P, che puo' venire omesso e rimosso dall'equazione, visto che in caso non ci siano misure di prevenzione dell'erosione il valore corrispondente al punto della matrice corrisponde a 1.

\subsection{R o erosività delle precipitazioni}
Il parametro R rappresenta quanto le precipitazioni in una determinata area geografica. Viene fatto attraverso dati pluviometrici. Questi ultimi non sono dati con risoluzione temporale pari a un singolo evento meteorico ma da equazioni semplificate per ottenere il parametro da i dati di precipitazioni annue.

\begin{equation}\label{eq:r}
	R = \sum_{i=1}^{n} \frac{p_i}{n}
\end{equation}
In questa equazione $n$ corrisponde il numero di anni di cui si ha lo storico pluviometrico mentre $p_i$ corrisponde alle precipitazioni dell'$i$-esimo anno. Questo permette una approssimazione utilile per il nostro problema.

\subsection{K o erodibilità del suolo}
Il parametro K rappresenta quanto il terreno puo' essere eroso. Questo dipende molto da come  strutturato il terreno, dalla sua composizione geologica e da alcune caratteristiche fisiche e chimiche dello stesso.

Questo tipo di dato non è disponibile per aree geografiche di grosse dimensioni per cui è stata fatta una regrassione matematica su dati sulla tessitura del terreno dell'area interessata. Questo a permesso la creazione di un modello matematico descritto dall'equazione di Romkens che permette la creazione di una stima precisa (per quanto riguarda i nostri scopi) dell'erodibilità del suolo.

\begin{equation}\label{eq:k}
	K=0,0034 + 0,0405 * \exp{-0,5\left( \dfrac{(\log(D_g)+1,659)^2}{0,7101} \right) }
\end{equation}

\subsection{LS rapporto tra lunghezza e pendenza}
LS rappresenza un rapporto tra lunghezza e pendenza a meno di due parametri $m=0,4$ e $n=1,3$ definiti per convenzione. 
\begin{equation}\label{eq:ls}
	LS = \left(  \dfrac{A}{22,13} \right)^m
	\left(  \dfrac{\sin(\alpha)}{0,0896} \right)^n
\end{equation}


\section{Possibili risoluzioni}
La tesi \cite{tesi:ambientale} propone quattro soluzioni per il problema dell'erosione annua del terreno.
Sono state tutte prese in esame prima di iniziare lo sviluppo del plugin e qui riportiamo vantaggi, svantaggi e problemi legati a queste soluzioni.

\subsection{Software di grafic design}
Attraverso i classici programmi di creazione di ambienti virtuali è possibile visualizzare le modifiche dell'ambiente in base ai dati raccolti. Questo permette di creare una rappresentazione realistica e molto dettagliata dell'ambiente preso in esame ma richiede una grossa quantità di tempo e notevoli sforzi, senza contare che necessità di un esperto di modellazione 3D che si occupi del lavoro. In oltre questo metodo non permette l'automatizzazione ma necessita comunque di una componente umana che realizzi il modello 3D. 

Per questo motivo non è stato preso in esame questa soluzione. Infatti la possibilità di calcolare e visualizzare l'erosione del terreno deve essere facilmente automatizzabile e i risultati forniti in tempi brevi.

\subsection{Strumenti di visualizzazione 3D}
E' possibile utilizzare strumenti di visualizzazione di modelli 3D specifici per la visualizzazione di ambienti reali. Questo permette la visualizzazione dell'area presa in esame confrontandola anche con il terreno circostante ma prevede che sia già stato generato un file nel formato leggibile dal programma scelto.

Questo ripropone i problemi dati dalla scelta di un software di grafic design in quanto prevede che ci sia un software o un "tecnico" che generi il file nel formato richiesto rendendo l'applicativo non automatizzabile.

\subsection{Programmazione}
Attraverso alcuni linguaggi di programmazione è possibile calcolare il raster di output. Per esempio è possibile farlo con Mathlab, R o Python con tempi di esecuzione diverso in base al linguaggio e a quanto è stato ottimizzato il codice che esegue queste operazioni. 

Questo risolve il problema dell'automatizzazione del processo, permettendo così di creare un servizio di elaborazione dei dati autonomo e indipendente dall'essere umano, ma non realizza l'altro punto importante del progetto: non permette di visualizzare i dati in input ne in output.

In oltre richiede un notevole lavoro in quanto questa soluzione necessita di scrivere un programma per intero, gestire le GUI e tutta la gestione dei formati di file solo per avere l'espressione \ref{eq:main} esequita.

\subsection{Desktop QGis}

Software che permette di visualizzare e elaborare raster e altri formati di file geomorfici. Molto usato e supporta, nativamente o attraverso plugin installabili, la quasi totalità dei formati di file esistenti. In oltre permette, attraverso codice python, c e c++, di estendere le funzionalità del programma stesso.

Il vantaggio fondamentale di questo sistema è l'estremo supporto di formati diversi e la presenza di un motore grafico che permette la visualizzazione corretta dei dati, con supporto con i formati di file per mappe e con i servizi online come google maps. La sua installazione, in oltre, contiene alcune delle librerie più complete per quanto riguarda le operazioni sui raster e array.

Invece comporta anche dei diffetti. Limita infatti sia i linguaggi\footnote{E' possibile sviluppare solo per Python, C e C++ i plugin anche se consigliano di usare Python per una questione di velocità di sviluppo} in cui è possibile sviluippare un plugin o estensione e ha solo alcune delle maggiori librerie diponibili.

\section{QGis}

QGis si definisce come "Un Sistema di Informazione Geografica Libero e Open Source"\cite{site:qgis} ovvero è un software che gestisce, elabora e visualizza dati geomorfici e georeferenziati, completamente gratuito e modificabile in ogni sua parte(è limitato dalla licenza utilizzata nello sviluppo del software \cite{site:cc3}).

In particolare sono state utilizzate le componenti di gestione input e output dei raster, il motore di calcolo e il sistema di rendering video. In oltre il programma supporta anche un sistema di plugin e scripting che aumenta le funzionalità del programma utilizzando codice esterno sviluppato dalla comunnity.

Altro dettaglio molto utile di QGis è il fatto che un programma multi piattaforma. Questo vuol dire che il programma è stato distribuito per più sistemi operativi\footnote{In questo caso per Android, Mac OS, Windows e Linux} e obbliga gli sviluppatori a scrivere plugin e script che siano anch'essi multipiattaforma. Questo permette l'utilizzo dei plugin e degli script per la creazione di funzioni automatizzate utilizzando le componenti già presenti in QGis o aggiungerne di nuove.

In oltre, QGis supporta la quasi totalità dei formati open source utilizzati per la codifica per dati geomorfici e georeferenziati in modo nativo mentre i formati di file che non supporta sono utilizzabili attraverso plugin gratuiti disponibili online.

\subsection{Descrizione dei tipi di dati supportati}

Per risolvere il problema presentato è necessario passare al programma i dati. Sono devono essere già presenti all'interno dell'elaboratore o possono essere letti da un flusso in ingresso proveniente da sensori o da altri elaboratori.

Per il nostro problema i dati vengono forniti sotto forma di file o di database in quanto possono necessetira di una pre elaborazione e una formattazione, e di conseguenza non è conveniente l'utilizzo di un flusso. In oltre è possibile che i dati non vengano da un unico sensore in quanto sono dati legati a composizione geologica e a piogge in determinate aree e, di conseguenza, che necessitano di multipli sensori.


I formati di dati possono essere divisi in tre macrocategorie in base a come vengono forniti:
\begin{itemize}
	\item Vettoriali
	\item Raster
	\item DBRMS con estensione spaziale
\end{itemize}

Questi tre formati permettono solo alcune operazioni possibili legate alla composizione dei dati nel formato di input.
Quindi, per alcune operazioni, QGis converte i dati nell formato appropiato in modo che sia sempre possibile fare tutte le operazioni disponibili, anche se queste puo' portare ad approssimazioni dei dati nella conversione.

\subsection{La struttura di QGis}
QGis è un programma avanzato con molte funzionalità che possono essere suddivise in alcune macro-categorie che sono indipendenti dal tipo di dato o file utilizzato.
Le funzionalità standard di QGis sono gestite attraverso dei moduli che costituiscono l'istallazione di base del programma e le funzionalità aggiuntive vengono implementate attraverso plugin dipendenti da questi moduli di base per poter funzionare correttamente.
Solitamente i moduli sono scritti in C o C++ mentre i plugin e gli script sono in Python.

\subsubsection{Rendering grafico dei dati}
QGis ha la capacità di realizzare svariate visualizzazioni grafiche dei dati inseriti o calcolati.
Questo permette la realizzazioni di immagini ad alta definizione dei dati inseriti e ne permetta anche l'esportazione nei formati più comuni di immagini.

\subsubsection{Realizzazione mappe}
In presenza di dati georeferenziati QGis permette di sovrapporre la rappresentazione grafica dei dati alla mappa corrispondente alle coordinate georeferenziali.
Questo permette di creare mappe fisiche contenenti la visualizzazione di aree di erosione attraverso aree colorate o altri effetti applicabili alla mappa sottostante.

\subsubsection{Creazione, elaborazione e conversione dati}
QGis presenta due moduli per l'elaborazioni di dati:
\begin{itemize}
	\item \textbf{Class} modulo standard di QGis per i calcoli aritmetici elementari e alcuni calcoli statistici di base
	\item \textbf{Grass} modulo aggiuntivo di QGis per i calcoli avanzati e regressioni. Contiene anche modelli previsionali e probabilistici.
\end{itemize}

\subsubsection{Analisi dati}
Quando si lavora con un insieme di dati non è detto che questi siano tutti corretti o non abbiano un livello di approssimazione.
All'interno di QGis sono disponibili, di base, un insieme di funzionalità atte a calcolare la bontà delle approssimazioni o dei dati stessi presi in esame dal programma.
Questo viene fatto di base con algoritmi generici per la maggior parte dei formati ma se si usa solo un tipo specifico o si vuole una approssimazione o un modello preciso è possibile ottenere il risultato desiderato attraverso plugin di terze parti.

\subsection{Sistema dei plugin}
Per aumentare lel funzionalità di QGis e scriptare alcune operazioni sono stati creati i plugin.
Sono dei programmi dipendenti da QGis che implementano funzionalità non presenti nel programma di base o estendono il supporto delle funzioni esistenti o i tipi di file supportati.

\subsubsection{Plugin per formati di file}
Vengono utilizzati per elaborare dati grezzi altrimenti difficilmente manipolabili.
Un esempio possono essere i dati grezzi ottenuti da degli strumenti di misurazione con dati in output non standard.
Normalmente questi file sono leggibili solo col programma abilitato alle modifiche ma, con l'utilizzo del plugin corretto, è possibile leggerlo e elaborarlo direttamente in QGis.

\subsubsection{Pluing che alterano o aumentano le funzionalità}
La reale forza di QGis. Permettono di ampliare le funzioni del programma semplificando o aumentando le funzionalità pre esisteni. Alcuni esempi sono i plugin della famiglia GRASS che aggiungono operazioni matematiche o gli strumenti di creazione timelapse partendo da una serie di ruster temporizzati.s
