% !TEX encoding = UTF-8
% !TEX TS-program = pdflatex
% !TEX root = ../Tesi.tex
% !TEX spellcheck = it-IT

%************************************************
\chapter{Descrizione approfondita dell'argomento}
\label{cap:descrizione}
%************************************************
\section{Problema da risolvere} Lo scopo di questa tesi è quello di calcolare in
modo corretto l'erosione del territorio nell'area geografica indicata. Questo è
stato fatto utilizzando un modello RUSLE basato sull'equazione\footnote{Il
modello proposto è quello indicato in \cite[p.~37]{tesi:ambientale}}

\begin{equation}
A = R * K * LS * C * P
\end{equation}

in cui
\begin{itemize}
\item \textbf{A} corrispondente alla perdita di suolo annua
\item \textbf{R} erosività delle precipitazioni
\item \textbf{K} erodibilità del suolo
\item \textbf{LS} rapporto lunghezza pendenza
\item \textbf{C} fattore di copertura del suolo
\item \textbf{P} misure di prevenzione dell'erosione
\end{itemize}

Questo modello è stato quindi applicato utilizzando i dati in formato di layer raster attraverso l'applicativo QGis.
Questo viene fatto in quanto lo standard \textit{de facto} per questa tipologia di dato è il layer raster.


In particolare questo modello viene utilizzato per calcolare anno per anno l'erosione del territorio interessato avendo variazioni di dati in base all'anno o assumendo come invarianti negli anni i dati inseriti in imput.

\section{QGis}

QGis si definisce come "Un Sistema di Informazione Geografica Libero e Open Source"\cite{site:qgis} ovvero è un software che gestisce, elabora e visualizza dati geomorfici e georeferenziati, completamente gratuito e modificabile in ogni sua parte(è limitato dalla licenza utilizzata nello sviluppo del software \cite{site:cc3}).

In particolare sono state utilizzate le componenti di gestione input e output dei raster, il motore di calcolo e il sistema di rendering video. In oltre il programma supporta anche un sistema di plugin e scripting che aumenta le funzionalità del programma utilizzando codice esterno sviluppato dalla comunnity. 

Altro dettaglio molto utile di QGis è il fatto che un programma multi piattaforma. Questo vuol dire che il programma è stato distribuito per più sistemi operativi\footnote{In questo caso per Android, Mac OS, Windows e Linux} e obbliga gli sviluppatori a scrivere plugin e script che siano anch'essi multipiattaforma. Questo permette l'utilizzo dei plugin e degli script per la creazione di funzioni automatizzate utilizzando le componenti già presenti in QGis o aggiungerne di nuove.

In oltre, QGis supporta la quasi totalità dei formati open source utilizzati per la codifica per dati geomorfici e georeferenziati in modo nativo mentre i formati di file che non supporta sono utilizzabili attraverso plugin gratuiti disponibili online.

\subsection{Descrizione dei tipi di dati supportati}

I formati di dati possono essere divisi in tre macrocategorie in base a come vengono forniti:
\begin{itemize}
	\item Vettoriali
	\item Raster
	\item DBRMS con estensione spaziale
\end{itemize}


\subsection{Architettura di QGis}

\subsection{Sistema dei plugin}
