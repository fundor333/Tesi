% !TEX encoding = UTF-8
% !TEX TS-program = pdflatex
% !TEX root = ../Tesi.tex
% !TEX spellcheck = it-IT

%************************************************
\chapter{Descrizione approfondita dell'argomento}
\label{cap:descrizione}
%************************************************
\section{Problema da risolvere} Lo scopo di questa tesi è quello di calcolare in
modo corretto l'erosione del territorio nell'area geografica scielta.
Questo è stato fatto utilizzando un modello RUSLE basato sull'equazione\footnote{Il
modello proposto è quello indicato in \cite[p.~37]{tesi:ambientale}}

\begin{equation} \label{eq:main}
A = R * K * LS * C * P
\end{equation}

in cui
\begin{itemize}
\item \textbf{A} corrispondente alla perdita di suolo annua
\item \textbf{R} erosività delle precipitazioni
\item \textbf{K} erodibilità del suolo
\item \textbf{LS} rapporto lunghezza pendenza
\item \textbf{C} fattore di copertura del suolo
\item \textbf{P} misure di prevenzione dell'erosione
\end{itemize}

Questo modello è stato quindi applicato utilizzando i dati in formato di layer raster attraverso l'applicativo QGis.
Questo viene fatto in quanto lo standard \textit{de facto} per questa tipologia di dato è il layer raster.


In particolare questo modello viene utilizzato per calcolare anno per anno l'erosione del territorio interessato avendo variazioni di dati in base all'anno o assumendo come invarianti negli anni i dati inseriti in imput.

\section{QGis}

QGis si definisce come "Un Sistema di Informazione Geografica Libero e Open Source"\cite{site:qgis} ovvero è un software che gestisce, elabora e visualizza dati geomorfici e georeferenziati, completamente gratuito e modificabile in ogni sua parte(è limitato dalla licenza utilizzata nello sviluppo del software \cite{site:cc3}).

In particolare sono state utilizzate le componenti di gestione input e output dei raster, il motore di calcolo e il sistema di rendering video. In oltre il programma supporta anche un sistema di plugin e scripting che aumenta le funzionalità del programma utilizzando codice esterno sviluppato dalla comunnity.

Altro dettaglio molto utile di QGis è il fatto che un programma multi piattaforma. Questo vuol dire che il programma è stato distribuito per più sistemi operativi\footnote{In questo caso per Android, Mac OS, Windows e Linux} e obbliga gli sviluppatori a scrivere plugin e script che siano anch'essi multipiattaforma. Questo permette l'utilizzo dei plugin e degli script per la creazione di funzioni automatizzate utilizzando le componenti già presenti in QGis o aggiungerne di nuove.

In oltre, QGis supporta la quasi totalità dei formati open source utilizzati per la codifica per dati geomorfici e georeferenziati in modo nativo mentre i formati di file che non supporta sono utilizzabili attraverso plugin gratuiti disponibili online.

\subsection{Descrizione dei tipi di dati supportati}

I formati di dati possono essere divisi in tre macrocategorie in base a come vengono forniti:
\begin{itemize}
	\item Vettoriali
	\item Raster
	\item DBRMS con estensione spaziale
\end{itemize}

Questi tre formati permettono solo alcune operazioni possibili legate alla composizione dei dati nel formato di input.
Quindi, per alcune operazioni, QGis converte i dati nell formato appropiato in modo che sia sempre possibile fare tutte le operazioni disponibili, anche se queste puo' portare ad approssimazioni dei dati nella conversione.

\subsection{La struttura di QGis}
QGis è un programma avanzato con molte funzionalità che possono essere suddivise in alcune macro-categorie che sono indipendenti dal tipo di dato o file utilizzato.
Le funzionalità standard di QGis sono gestite attraverso dei moduli che costituiscono l'istallazione di base del programma e le funzionalità aggiuntive vengono implementate attraverso plugin dipendenti da questi moduli di base per poter funzionare correttamente.
Solitamente i moduli sono scritti in C o C++ mentre i plugin e gli script sono in Python.

\subsubsection{Rendering grafico dei dati}
QGis ha la capacità di realizzare svariate visualizzazioni grafiche dei dati inseriti o calcolati.
Questo permette la realizzazioni di immagini ad alta definizione dei dati inseriti e ne permetta anche l'esportazione nei formati più comuni di immagini.

\subsubsection{Realizzazione mappe}
In presenza di dati georeferenziati QGis permette di sovrapporre la rappresentazione grafica dei dati alla mappa corrispondente alle coordinate georeferenziali.
Questo permette di creare mappe fisiche contenenti la visualizzazione di aree di erosione attraverso aree colorate o altri effetti applicabili alla mappa sottostante.

\subsubsection{Creazione, elaborazione e conversione dati}
QGis presenta due moduli per l'elaborazioni di dati:
\begin{itemize}
	\item \textbf{Class} modulo standard di QGis per i calcoli aritmetici elementari e alcuni calcoli statistici di base
	\item \textbf{Grass} modulo aggiuntivo di QGis per i calcoli avanzati e regressioni. Contiene anche modelli previsionali e probabilistici.
\end{itemize}

\subsubsection{Analisi dati}
Quando si lavora con un insieme di dati non è detto che questi siano tutti corretti o non abbiano un livello di approssimazione.
All'interno di QGis sono disponibili, di base, un insieme di funzionalità atte a calcolare la bontà delle approssimazioni o dei dati stessi presi in esame dal programma.
Questo viene fatto di base con algoritmi generici per la maggior parte dei formati ma se si usa solo un tipo specifico o si vuole una approssimazione o un modello preciso è possibile ottenere il risultato desiderato attraverso plugin di terze parti.

\subsection{Sistema dei plugin}
Per aumentare lel funzionalità di QGis e scriptare alcune operazioni sono stati creati i plugin.
Sono dei programmi dipendenti da QGis che implementano funzionalità non presenti nel programma di base o estendono il supporto delle funzioni esistenti o i tipi di file supportati.

\subsubsection{Plugin per formati di file}
Vengono utilizzati per elaborare dati grezzi altrimenti difficilmente manipolabili.
Un esempio possono essere i dati grezzi ottenuti da degli strumenti di misurazione con dati in output non standard.
Normalmente questi file sono leggibili solo col programma abilitato alle modifiche ma, con l'utilizzo del plugin corretto, è possibile leggerlo e elaborarlo direttamente in QGis.

\subsubsection{Pluing che alterano o aumentano le funzionalità}
La reale forza di QGis. Permettono di ampliare le funzioni del programma semplificando o aumentando le funzionalità pre esisteni. Alcuni esempi sono i plugin della famiglia GRASS che aggiungono operazioni matematiche o gli strumenti di creazione timelapse partendo da una serie di ruster temporizzati.s
