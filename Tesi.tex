% !0TEX encoding = UTF-8 Unicode
% !TEX TS-program = pdflatex
% !TEX root = Tesi.tex
% !TeX spellcheck = it_IT

\documentclass[14pt,%                      % corpo del font principale
               a4paper,%                   % carta A4
               twoside,openright,%         % fronte-retro
%              oneside,openany,%           % solo fronte
               ]{book}
\usepackage[T1]{fontenc}                   % codifica dei font:
                                           % NOTA BENE! richiede una distribuzione *completa* di LaTeX
                                           % per esempio TeXLive o MiKTeX *complete*

\usepackage[utf8]{inputenc}                % codifica di input; anche [latin1] va bene
                                           % NOTA BENE! va accordata con le preferenze dell'editor

\usepackage{microtype}                     % microtipografia

\usepackage[english,italian]{babel}        % per scrivere in italiano e in inglese;
                                           % l'ultima lingua (l'italiano) risulta predefinita

\usepackage[binding=5mm]{layaureo}         % margini ottimizzati per l'A4; rilegatura di 5 mm

\usepackage[suftesi]{frontespizio}         % frontespizo
                                           % per includerlo nel documento bisogna:
                                           % 1. compilare una prima volta Tesi.tex;
                                           % 2. compilare a parte Tesi-frn.tex, generato dalla compilazione precedente;
                                           % 3. compilare ancora Tesi.tex.

\usepackage{emptypage}                     % pagine vuote senza testatina e piede di pagina

\usepackage{indentfirst}                   % rientra il primo capoverso di ogni sezione

\usepackage{booktabs}                      % tabelle

\usepackage{tabularx}                      % tabelle di larghezza prefissata

\usepackage{graphicx}                      % immagini

\usepackage{subfig}                        % sottofigure, sottotabelle

\usepackage{caption}                       % didascalie

\usepackage{listings}                      % codici

\usepackage[font=small]{quoting}           % citazioni

\usepackage{amsmath,amssymb,amsthm}        % matematica

\usepackage[italian]{varioref}             % riferimenti completi della pagina

\usepackage{mparhack,fixltx2e,relsize}     % finezze tipografiche

\usepackage[style=alphabetic,backend=bibtex]{biblatex}
                                           % eccellente pacchetto per la bibliografia;
                                           % produce uno stile di citazione autore-anno;
                                           % lo stile "numeric-comp" produce riferimenti numerici

\bibliography{Bibliografia}

\usepackage{chngpage,calc}                 % centra il frontespizio

\usepackage[dvipsnames]{xcolor}            % colori

\usepackage{lipsum}                        % testo fittizio

\usepackage{eurosym}                       % simbolo dell'euro

\usepackage{hyperref}                      % collegamenti ipertestuali

\usepackage{bookmark}                      % segnalibri

%*********************************************************************************
% Comandi persaonali
%*******************************************************
\newcommand{\myName}{Lorenzo Pantieri}                       % autore
\newcommand{\myTitle}{La prova ontologica dell'esistenza di Dio} % titolo
\newcommand{\myDegree}{Tesi di laurea}                       % tipo di tesi
\newcommand{\myUni}{Universit\`a degli Studi del Magdeburgo} % universit\`a
\newcommand{\myFaculty}{Facolt\`a di Lettere e Filosofia}    % facolt\`a
\newcommand{\myDepartment}{Dipartimento di Teologia}         % dipartimento
\newcommand{\myProf}{Chiar.mo Prof.~S.~Anselmo d'Aosta}      % relatore
%\newcommand{\myOtherProf}{Dott.~Immanuel Kant}              % eventuale correlatore
\newcommand{\myLocation}{Magdeburgo}                         % dove
\newcommand{\myTime}{Dicembre 2011}                          % quando



%*********************************************************************************
% Impostazioni di amsmath, amssymb, amsthm
%*********************************************************************************

% comandi per gli insiemi numerici (serve il pacchetto amssymb)
\newcommand{\numberset}{\mathbb}
\newcommand{\N}{\numberset{N}}
\newcommand{\R}{\numberset{R}}
\newcommand{\rusle}{\textbf{\textit{Rusle}}}

% un ambiente per i sistemi
\newenvironment{sistema}%
  {\left\lbrace\begin{array}{@{}l@{}}}%
  {\end{array}\right.}

% definizioni (serve il pacchetto amsthm)
\theoremstyle{definition}
\newtheorem{definizione}{Definizione}

% teoremi, leggi e decreti (serve il pacchetto amsthm)
\theoremstyle{plain}
\newtheorem{teorema}{Teorema}
\newtheorem{legge}{Legge}
\newtheorem{dimostrazione}{Dimostrazione}
\newtheorem{decreto}[legge]{Decreto}
\newtheorem{murphy}{Murphy}[section]



%*********************************************************************************
% Impostazioni di biblatex
%*********************************************************************************
\defbibheading{bibliography}{%
\cleardoublepage
\phantomsection
\addcontentsline{toc}{chapter}{\bibname}
\chapter*{\bibname\markboth{\bibname}
{\bibname}}}


%*********************************************************************************
% Impostazioni di listings
%*********************************************************************************
\lstset{language=[LaTeX]Tex,%C++,
    keywordstyle=\color{RoyalBlue},%\bfseries,
    basicstyle=\small\ttfamily,
    %identifierstyle=\color{NavyBlue},
    commentstyle=\color{Green}\ttfamily,
    stringstyle=\rmfamily,
    numbers=none,%left,%
    numberstyle=\scriptsize,%\tiny
    stepnumber=5,
    numbersep=8pt,
    showstringspaces=false,
    breaklines=true,
    frameround=ftff,
    frame=single
}



%*********************************************************************************
% Impostazioni di hyperref
%*********************************************************************************
\hypersetup{%
    hyperfootnotes=false,pdfpagelabels,
    %draft,	% = elimina tutti i link (utile per stampe in bianco e nero)
    colorlinks=true, linktocpage=true, pdfstartpage=1, pdfstartview=FitV,%
    % decommenta la riga seguente per avere link in nero (per esempio per la stampa in bianco e nero)
    %colorlinks=false, linktocpage=false, pdfborder={0 0 0}, pdfstartpage=1, pdfstartview=FitV,%
    breaklinks=true, pdfpagemode=UseNone, pageanchor=true, pdfpagemode=UseOutlines,%
    plainpages=false, bookmarksnumbered, bookmarksopen=true, bookmarksopenlevel=1,%
    hypertexnames=true, pdfhighlight=/O,%nesting=true,%frenchlinks,%
    urlcolor=webbrown, linkcolor=RoyalBlue, citecolor=webgreen, %pagecolor=RoyalBlue,%
    %urlcolor=Black, linkcolor=Black, citecolor=Black, %pagecolor=Black,%
    pdftitle={\myTitle},%
    pdfauthor={\textcopyright\ \myName, \myUni, \myFaculty},%
    pdfsubject={},%
    pdfkeywords={},%
    pdfcreator={pdfLaTeX},%
    pdfproducer={LaTeX with hyperref and ClassicThesis}%
}


%*********************************************************************************
% Impostazioni di graphicx
%*********************************************************************************
\graphicspath{{Immagini/}} % cartella dove sono riposte le immagini

%*********************************************************************************
% Impostazioni di xcolor
%*********************************************************************************
\definecolor{webgreen}{rgb}{0,.5,0}
\definecolor{webbrown}{rgb}{.6,0,0}

%*********************************************************************************
% Impostazioni di caption
%*********************************************************************************
\captionsetup{tableposition=top,figureposition=bottom,font=small,format=hang,labelfont=bf}

%*********************************************************************************
% Altro
%*********************************************************************************

% [...] ;-)
\newcommand{\omissis}{[\dots\negthinspace]}

% eccezioni all'algoritmo di sillabazione
\hyphenation{Fortran ma-cro-istru-zio-ne nitro-idrossil-amminico}
                  % file con le impostazioni personali

\begin{document}


\frontmatter
%******************************************************************
% Materiale iniziale
%******************************************************************
% !TEX encoding = UTF-8
% !TEX TS-program = pdflatex
% !TEX root = ../Tesi.tex
% !TEX spellcheck = it-IT

%*******************************************************
% Frontespizio
%*******************************************************
\begin{frontespizio}
\Universita{Venezia}
\Logo{unive_logo.jpg}
\Dipartimento{Scienza Ambientali, Informatiche e Statistiche}
\Corso[Laurea]{Informatica}
\Annoaccademico{2016--2017}
\Titoletto{Tesi di laurea}
\Titolo{Implementazione del modello RUSLE attraverso plugin per QGis}
\Sottotitolo{}
\Candidato[845087]{Matteo Scarpa}
\Relatore{Claudio Silvestri}
\end{frontespizio}





%*******************************************************
% Frontespizio alternativo
%*******************************************************
%\begin{titlepage}
%\pdfbookmark{Frontespizio}{Frontespizio}
%\changetext{}{}{}{((\paperwidth - \textwidth) / 2) - \oddsidemargin - \hoffset - 1in}{}
%\null\vfill
%\begin{center}
%\large
%\sffamily
%\bigskip

%{\LARGE\myName} \\

%\bigskip

%{\Huge\myTitle \\
%}

%\bigskip
    
%\vspace{9cm}

%\begin{tabular}{cc}
%\parbox{0.3\textwidth}{\includegraphics[width=2.5cm]{Sigillo}}
%&
%\parbox{0.7\textwidth}{{\Large\myDegree} \\ 

%					{\normalsize
%					Relatore: \myProf \\
%%					Co-relatore: \myOtherProf \\
%					
%					\myUni \\
%					\myFaculty \\
%					\myDepartment \\
%					\myTime}}
%			\end{tabular}
%\end{center}
%\vfill
%\end{titlepage}
% !TEX encoding = UTF-8
% !TEX TS-program = pdflatex
% !TEX root = ../Tesi.tex
% !TEX spellcheck = it-IT

%*******************************************************
% Colophon
%*******************************************************
\clearpage
\phantomsection
\thispagestyle{empty}

\hfill

\vfill

%\noindent\myName: \textit{\myTitle,}
%\myDegree,
%\textcopyright\ \myTime.

\lipsum[2]
% !TEX encoding = UTF-8
% !TEX TS-program = pdflatex
% !TEX root = ../Tesi.tex
% !TEX spellcheck = it-IT

%*******************************************************
% Dedica
%*******************************************************
\cleardoublepage
\phantomsection
\thispagestyle{empty}
\pdfbookmark{Dedica}{Dedica}

\vspace*{3cm}

\begin{center}
Dedicata a tutti quelli che mi sono stati vicini nelle giornate no.
\end{center}

% !TEX encoding = UTF-8
% !TEX TS-program = pdflatex
% !TEX root = ../Tesi.tex
% !TEX spellcheck = it-IT

%*******************************************************
% Indici
%*******************************************************
\cleardoublepage
\pdfbookmark{\contentsname}{tableofcontents}
\setcounter{tocdepth}{2}
\tableofcontents
%\markboth{\contentsname}{\contentsname} 
\clearpage

%
%\begingroup 
%    \let\clearpage\relax
%    \let\cleardoublepage\relax
%    \let\cleardoublepage\relax
    %*******************************************************
    % Elenco delle figure
    %*******************************************************    
%    \phantomsection
%    \pdfbookmark{\listfigurename}{lof}
%    \listoffigures

%    \vspace*{8ex}

    %*******************************************************
    % Elenco delle tabelle
    %*******************************************************
%    \phantomsection
%    \pdfbookmark{\listtablename}{lot}
%    \listoftables
        
%    \vspace*{8ex}
       
%\endgroup

\cleardoublepage

%TODO % !TEX encoding = UTF-8
% !TEX TS-program = pdflatex
% !TEX root = ../Tesi.tex
% !TEX spellcheck = it-IT

%*******************************************************
% Sommario+Abstract
%*******************************************************
\cleardoublepage
\phantomsection
\pdfbookmark{Sommario}{Sommario}
\begingroup
\let\clearpage\relax
\let\cleardoublepage\relax
\let\cleardoublepage\relax

\chapter*{Sommario}

\lipsum[1]

\vfill

\selectlanguage{english}
\pdfbookmark{Abstract}{Abstract}
\chapter*{Abstract}

\lipsum[2]

\selectlanguage{italian}

\endgroup			

\vfill


%TODO % !TEX encoding = UTF-8
% !TEX TS-program = pdflatex
% !TEX root = ../Tesi.tex
% !TEX spellcheck = it-IT

%*******************************************************
% Ringraziamenti
%*******************************************************
\cleardoublepage
\phantomsection
\pdfbookmark{Ringraziamenti}{ringraziamenti}

\begin{flushright}{\slshape    
	Lorem ipsum dolor sit amet, consectetuer adipiscing elit. \\
	Ut purus elit, vestibulum ut, placerat ac, adipiscing vitae, felis. \\
	Curabitur dictum gravida mauris.} \\ \medskip
    --- Donald Ervin Knuth
\end{flushright}


\bigskip

\begingroup
\let\clearpage\relax
\let\cleardoublepage\relax
\let\cleardoublepage\relax

\chapter*{Ringraziamenti}

\lipsum[1]

\bigskip
 
\noindent\textit{\myLocation, \myTime}
\hfill L.~P.

\endgroup


\cleardoublepage
%******************************************************************
% Materiale principale
%******************************************************************
\mainmatter
% !TEX encoding = UTF-8
% !TEX TS-program = pdflatex
% !TEX root = ../Tesi.tex
% !TEX spellcheck = it-IT

%*******************************************************
% Introduzione
%*******************************************************
\cleardoublepage
\pdfbookmark{Introduzione}{introduzione}

\chapter*{Introduzione}

In questa tesi viene spiegato come è stato realizzato un plugin per QGis per il calcolo del terreno eroso avendo dei dati geomorfici legati al territorio scelto. Questi dati comprendono la conformazione del terreno, le precipitazioni registrate nell'area, il tipo di copertura della vegetazione e il numero di anni per cui questa previsione viene eseguita.

Questo plugin è stato scritto in Python e implementa il plugin ipotizzato nella tesi \cite{tesi:ambientale} che risulta essere indipendente e utilizzabile per il calcolo su una qualunque area geografica di cui si hanno i dati richiesti. In oltre il sistema risulta indipendente dalle dimensioni dell'area geografica selezionata. E' infatti possibile inserire un'area geografica di una qualsiasi dimensione per procedere con l'elaborazione. Da questo punto di vista il programma può procedere con aree geografiche grandi quanto continenti ma non è detto che il modello matematico si comporti correttamente con aree sufficientemente grandi o piccole rispetto alla precisione dei dati acquisiti.
% !TEX encoding = UTF-8
% !TEX TS-program = pdflatex
% !TEX root = ../Tesi.tex
% !TEX spellcheck = it-IT

%************************************************
\chapter{Descrizione dell'argomento}
\label{cap:descrizione}
%************************************************
\section{Problema da risolvere} Lo scopo di questa tesi è quello di calcolare in
modo corretto l'erosione del territorio nell'area geografica scielta.
Questo è stato fatto utilizzando un modello RUSLE basato sull'equazione\footnote{Il
modello proposto è quello indicato in \cite[p.~37]{tesi:ambientale}}

\begin{equation} \label{eq:main}
A = R * K * LS * C * P
\end{equation}

in cui
\begin{itemize}
\item \textbf{A} corrispondente alla perdita di suolo annua
\item \textbf{R} erosività delle precipitazioni
\item \textbf{K} erodibilità del suolo
\item \textbf{LS} rapporto lunghezza pendenza
\item \textbf{C} fattore di copertura del suolo
\item \textbf{P} misure di prevenzione dell'erosione
\end{itemize}

Questo modello è stato quindi applicato utilizzando i dati in formato di layer raster attraverso l'applicativo QGis.
Questo viene fatto in quanto lo standard \textit{de facto} per questa tipologia di dato è il layer raster.


In particolare questo modello viene utilizzato per calcolare anno per anno l'erosione del territorio interessato avendo variazioni di dati in base all'anno o assumendo come invarianti negli anni i dati inseriti in imput.

\section{QGis}

QGis si definisce come "Un Sistema di Informazione Geografica Libero e Open Source"\cite{site:qgis} ovvero è un software che gestisce, elabora e visualizza dati geomorfici e georeferenziati, completamente gratuito e modificabile in ogni sua parte(è limitato dalla licenza utilizzata nello sviluppo del software \cite{site:cc3}).

In particolare sono state utilizzate le componenti di gestione input e output dei raster, il motore di calcolo e il sistema di rendering video. In oltre il programma supporta anche un sistema di plugin e scripting che aumenta le funzionalità del programma utilizzando codice esterno sviluppato dalla comunnity.

Altro dettaglio molto utile di QGis è il fatto che un programma multi piattaforma. Questo vuol dire che il programma è stato distribuito per più sistemi operativi\footnote{In questo caso per Android, Mac OS, Windows e Linux} e obbliga gli sviluppatori a scrivere plugin e script che siano anch'essi multipiattaforma. Questo permette l'utilizzo dei plugin e degli script per la creazione di funzioni automatizzate utilizzando le componenti già presenti in QGis o aggiungerne di nuove.

In oltre, QGis supporta la quasi totalità dei formati open source utilizzati per la codifica per dati geomorfici e georeferenziati in modo nativo mentre i formati di file che non supporta sono utilizzabili attraverso plugin gratuiti disponibili online.

\subsection{Descrizione dei tipi di dati supportati}

I formati di dati possono essere divisi in tre macrocategorie in base a come vengono forniti:
\begin{itemize}
	\item Vettoriali
	\item Raster
	\item DBRMS con estensione spaziale
\end{itemize}

Questi tre formati permettono solo alcune operazioni possibili legate alla composizione dei dati nel formato di input.
Quindi, per alcune operazioni, QGis converte i dati nell formato appropiato in modo che sia sempre possibile fare tutte le operazioni disponibili, anche se queste puo' portare ad approssimazioni dei dati nella conversione.

\subsection{La struttura di QGis}
QGis è un programma avanzato con molte funzionalità che possono essere suddivise in alcune macro-categorie che sono indipendenti dal tipo di dato o file utilizzato.
Le funzionalità standard di QGis sono gestite attraverso dei moduli che costituiscono l'istallazione di base del programma e le funzionalità aggiuntive vengono implementate attraverso plugin dipendenti da questi moduli di base per poter funzionare correttamente.
Solitamente i moduli sono scritti in C o C++ mentre i plugin e gli script sono in Python.

\subsubsection{Rendering grafico dei dati}
QGis ha la capacità di realizzare svariate visualizzazioni grafiche dei dati inseriti o calcolati.
Questo permette la realizzazioni di immagini ad alta definizione dei dati inseriti e ne permetta anche l'esportazione nei formati più comuni di immagini.

\subsubsection{Realizzazione mappe}
In presenza di dati georeferenziati QGis permette di sovrapporre la rappresentazione grafica dei dati alla mappa corrispondente alle coordinate georeferenziali.
Questo permette di creare mappe fisiche contenenti la visualizzazione di aree di erosione attraverso aree colorate o altri effetti applicabili alla mappa sottostante.

\subsubsection{Creazione, elaborazione e conversione dati}
QGis presenta due moduli per l'elaborazioni di dati:
\begin{itemize}
	\item \textbf{Class} modulo standard di QGis per i calcoli aritmetici elementari e alcuni calcoli statistici di base
	\item \textbf{Grass} modulo aggiuntivo di QGis per i calcoli avanzati e regressioni. Contiene anche modelli previsionali e probabilistici.
\end{itemize}

\subsubsection{Analisi dati}
Quando si lavora con un insieme di dati non è detto che questi siano tutti corretti o non abbiano un livello di approssimazione.
All'interno di QGis sono disponibili, di base, un insieme di funzionalità atte a calcolare la bontà delle approssimazioni o dei dati stessi presi in esame dal programma.
Questo viene fatto di base con algoritmi generici per la maggior parte dei formati ma se si usa solo un tipo specifico o si vuole una approssimazione o un modello preciso è possibile ottenere il risultato desiderato attraverso plugin di terze parti.

\subsection{Sistema dei plugin}
Per aumentare lel funzionalità di QGis e scriptare alcune operazioni sono stati creati i plugin.
Sono dei programmi dipendenti da QGis che implementano funzionalità non presenti nel programma di base o estendono il supporto delle funzioni esistenti o i tipi di file supportati.

\subsubsection{Plugin per formati di file}
Vengono utilizzati per elaborare dati grezzi altrimenti difficilmente manipolabili.
Un esempio possono essere i dati grezzi ottenuti da degli strumenti di misurazione con dati in output non standard.
Normalmente questi file sono leggibili solo col programma abilitato alle modifiche ma, con l'utilizzo del plugin corretto, è possibile leggerlo e elaborarlo direttamente in QGis.

\subsubsection{Pluing che alterano o aumentano le funzionalità}
La reale forza di QGis. Permettono di ampliare le funzioni del programma semplificando o aumentando le funzionalità pre esisteni. Alcuni esempi sono i plugin della famiglia GRASS che aggiungono operazioni matematiche o gli strumenti di creazione timelapse partendo da una serie di ruster temporizzati.s

% !TEX encoding = UTF-8
% !TEX TS-program = pdflatex
% !TEX root = ../Tesi.tex
% !TEX spellcheck = it-IT

%************************************************
\chapter{Realizzazione}
\label{cap:realizzazione}

Il progetto è stato realizzato attraverso la divisione in subproblemi. Questo permette la sostituzione di una componente con un'altra in modo rapido ed sicuro in quanto viene a essere modificata solo la parte del codice interessata dall'operazione e non tutte le altre componenti, che rimangono invariate. 

Questo ha portato anche alla ricerca di alcune librerie per eseguire le operazioni necessarie alle funzioni scelte.

\section{Divisione in sottoproblemi}
Per semplificare lo sviluppo di questa applicazione e il suo sviluppo è stato diviso il progetto in quattro componenti indipendenti che dialogano tra di loro, in modo da rendere più isolate le varie componenti, rendendole meno soggette a errori non previsti. 
Queste si dividono in:
\begin{itemize}
\item Interfaccia grafica
\item Lettura dei file in ingresso
\item Gestione ed elaborazione dei dati
\item Scrittura dei file in uscita
\end{itemize}

Questa strutturazione permette anche il riciclo del codice utilizzato per risolvere i sottoproblemi in altri progetti con la stessa necessità o la facile estensione del plugin in un secondo momento

\subsection{Interfaccia grafica}
Al momento della progettazione dell'applicazione è stato necessario pensare come realizzare l'inserimento dei dati da elaborare da parte dell'utente. Questo, solitamente, viene fatto in due modi:
\begin{itemize}

\item Attraverso l'uso di un \textbf{terminale} in cui vengono passati dei comandi che eseguono le operazioni richieste e mostra a schermo tutti i dati dei comandi eseguiti. Questo però obbliga l'utente a sapere i comandi necessari alla esecuzione dei task e ritorna a schermo dati di poco interesse per l'utente medio, molto utile invece se sei un programmatore e vuoi controllare come procedono i vari passi dei comandi eseguiti e permette di vedere subito ove avvengono gli errori e cosa li generano. 

\item Attraverso l'uso di una \textbf{interfaccia grafica o GUI} in cui viene selezionato, attraverso l'uso di bottoni, menu e interuttori le funzioni che si vogliono eseguire e si danno i file in input con un selettore grafico. Permette all'utente di non interessarsi a come è stato realizzata l'operazione richiesta, di non cunsultare guide e manuali con gli elenchi dei comandi disponibili e di ricevere a schermo solo le informazioni di cui è realmente interessato. Oltre a essere la scelta più pratica per un nuovo utente è anche quella che da al programmatore che sviluppa questa applicazione la maggior libertà e indipendenza. Infatti la presenza di una grafica comune tra aggiornamenti diversi permette al programmatore di riscrivere interamente i comandi che vengono eseguiti al di sotto dell'interfaccia senza dover avvisare l'utente del cambiamento dei comandi all'interno del terminale.

\end{itemize}

Per come è strutturato QGis è possibile avere entrambi implementati nel plugin ma è stato preferito dare maggiore attenzione alla componente grafica anche se è comunque possibile aprire il terminale di QGis ed eseguire dei comandi. 

In particolare in questa applicazione è stata implementata l'interfaccia grafica in quanto risulta molto più adatta all'uso effettivo dell'applicazione. 

% TODO inserire qui immagine



\subsubsection{Qt4}
\subsection{Gestione input}
\subsection{Gestione dati}
\subsubsection{Grass}
\subsubsection{Osgeo}
\subsection{Gestione output}
\section{Librerie scelte}


\appendix
% !TEX encoding = UTF-8
% !TEX TS-program = pdflatex
% !TEX root = ../Tesi.tex
% !TEX spellcheck = it-IT

%************************************************

\chapter{Estendere le funzionalità del plugin}
\label{cap:appendice}
%************************************************

\section{Struttura del plugin}


% \appendix
% % !TEX encoding = UTF-8
% !TEX TS-program = pdflatex
% !TEX root = ../Tesi.tex
% !TEX spellcheck = it-IT

%************************************************

\chapter{Estendere le funzionalità del plugin}
\label{cap:appendice}
%************************************************

\section{Struttura del plugin}


% *****************************************************************
% Materiale finale
%******************************************************************
\backmatter
% !TEX encoding = UTF-8
% !TEX TS-program = pdflatex
% !TEX root = ../Tesi.tex
% !TEX spellcheck = it-IT

%*******************************************************
% Bibliografia
%*******************************************************
\cleardoublepage
\nocite{*}
\printbibliography
% % !TEX encoding = UTF-8
% !TEX TS-program = pdflatex
% !TEX root = ../Tesi.tex
% !TEX spellcheck = it-IT

%*******************************************************
% Dichiarazione
%*******************************************************
\cleardoublepage
\phantomsection
\pdfbookmark{Dichiarazione}{Dichiarazione}
\chapter*{Dichiarazione}
\thispagestyle{empty}
Lorem ipsum dolor sit amet, consectetuer adipiscing elit. Ut purus elit, vestibulum ut, placerat ac, adipiscing vitae, felis. Curabitur dictum gravida mauris. Nam arcu libero, nonummy eget, consectetuer id, vulputate a, magna. Donec vehicula augue eu neque.

Pellentesque habitant morbi tristique senectus et netus et malesuada fames ac turpis egestas. Mauris ut leo. Cras viverra metus rhoncus sem. Nulla et lectus vestibulum urna fringilla ultrices.

\bigskip
 
\noindent\textit{\myLocation, \myTime}

\smallskip

\begin{flushright}
    \begin{tabular}{m{5cm}}
        \\ \hline
        \centering\myName \\
    \end{tabular}
\end{flushright}

\end{document}
