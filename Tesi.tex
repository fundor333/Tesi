% !0TEX encoding = UTF-8 Unicode
% !TEX TS-program = pdflatex
% !TEX root = Tesi.tex
% !TeX spellcheck = it_IT

\documentclass[14pt,%                      % corpo del font principale
               a4paper,%                   % carta A4
               twoside,openright,%         % fronte-retro
%              oneside,openany,%           % solo fronte
               ]{book}
\usepackage[T1]{fontenc}                   % codifica dei font:
                                           % NOTA BENE! richiede una distribuzione *completa* di LaTeX
                                           % per esempio TeXLive o MiKTeX *complete*

\usepackage[utf8]{inputenc}                % codifica di input; anche [latin1] va bene
                                           % NOTA BENE! va accordata con le preferenze dell'editor

\usepackage{microtype}                     % microtipografia

\usepackage[english,italian]{babel}        % per scrivere in italiano e in inglese;
                                           % l'ultima lingua (l'italiano) risulta predefinita

\usepackage[binding=5mm]{layaureo}         % margini ottimizzati per l'A4; rilegatura di 5 mm

\usepackage[suftesi]{frontespizio}         % frontespizo
                                           % per includerlo nel documento bisogna:
                                           % 1. compilare una prima volta Tesi.tex;
                                           % 2. compilare a parte Tesi-frn.tex, generato dalla compilazione precedente;
                                           % 3. compilare ancora Tesi.tex.

\usepackage{emptypage}                     % pagine vuote senza testatina e piede di pagina

\usepackage{indentfirst}                   % rientra il primo capoverso di ogni sezione

\usepackage{booktabs}                      % tabelle

\usepackage{tabularx}                      % tabelle di larghezza prefissata

\usepackage{graphicx}                      % immagini

\usepackage{subfig}                        % sottofigure, sottotabelle

\usepackage{caption}                       % didascalie

\usepackage{listings}                      % codici

\usepackage[font=small]{quoting}           % citazioni

\usepackage{amsmath,amssymb,amsthm}        % matematica

\usepackage[italian]{varioref}             % riferimenti completi della pagina

\usepackage{mparhack,fixltx2e,relsize}     % finezze tipografiche

\usepackage[style=alphabetic,backend=bibtex]{biblatex}
                                           % eccellente pacchetto per la bibliografia;
                                           % produce uno stile di citazione autore-anno;
                                           % lo stile "numeric-comp" produce riferimenti numerici

\bibliography{Bibliografia}

\usepackage{chngpage,calc}                 % centra il frontespizio

\usepackage[dvipsnames]{xcolor}            % colori

\usepackage{lipsum}                        % testo fittizio

\usepackage{eurosym}                       % simbolo dell'euro

\usepackage{hyperref}                      % collegamenti ipertestuali

\usepackage{bookmark}                      % segnalibri

\input{impostazioni-tesi}                  % file con le impostazioni personali

\begin{document}


\frontmatter
%******************************************************************
% Materiale iniziale
%******************************************************************
% !TEX encoding = UTF-8
% !TEX TS-program = pdflatex
% !TEX root = ../Tesi.tex
% !TEX spellcheck = it-IT

%*******************************************************
% Frontespizio
%*******************************************************
\begin{frontespizio}
\Universita{Venezia}
\Logo{unive_logo.jpg}
\Dipartimento{Scienza Ambientali, Informatiche e Statistiche}
\Corso[Laurea]{Informatica}
\Annoaccademico{2016--2017}
\Titoletto{Tesi di laurea}
\Titolo{Implementazione del modello RUSLE attraverso plugin per QGis}
\Sottotitolo{}
\Candidato[845087]{Matteo Scarpa}
\Relatore{Claudio Silvestri}
\end{frontespizio}





%*******************************************************
% Frontespizio alternativo
%*******************************************************
%\begin{titlepage}
%\pdfbookmark{Frontespizio}{Frontespizio}
%\changetext{}{}{}{((\paperwidth - \textwidth) / 2) - \oddsidemargin - \hoffset - 1in}{}
%\null\vfill
%\begin{center}
%\large
%\sffamily
%\bigskip

%{\LARGE\myName} \\

%\bigskip

%{\Huge\myTitle \\
%}

%\bigskip
    
%\vspace{9cm}

%\begin{tabular}{cc}
%\parbox{0.3\textwidth}{\includegraphics[width=2.5cm]{Sigillo}}
%&
%\parbox{0.7\textwidth}{{\Large\myDegree} \\ 

%					{\normalsize
%					Relatore: \myProf \\
%%					Co-relatore: \myOtherProf \\
%					
%					\myUni \\
%					\myFaculty \\
%					\myDepartment \\
%					\myTime}}
%			\end{tabular}
%\end{center}
%\vfill
%\end{titlepage}
\input{MaterialeInizialeFinale/Colophon}
% !TEX encoding = UTF-8
% !TEX TS-program = pdflatex
% !TEX root = ../Tesi.tex
% !TEX spellcheck = it-IT

%*******************************************************
% Dedica
%*******************************************************
\cleardoublepage
\phantomsection
\thispagestyle{empty}
\pdfbookmark{Dedica}{Dedica}

\vspace*{3cm}

\begin{center}
Dedicata a tutti quelli che mi sono stati vicini nelle giornate no.
\end{center}

\input{MaterialeInizialeFinale/Indici}
%TODO \input{MaterialeInizialeFinale/Sommario+Abstract}
%TODO \input{MaterialeInizialeFinale/Ringraziamenti}
\cleardoublepage
%******************************************************************
% Materiale principale
%******************************************************************
\mainmatter
% !TEX encoding = UTF-8
% !TEX TS-program = pdflatex
% !TEX root = ../Tesi.tex
% !TEX spellcheck = it-IT

%*******************************************************
% Introduzione
%*******************************************************
\cleardoublepage
\pdfbookmark{Introduzione}{introduzione}

\chapter*{Introduzione}

In questa tesi viene spiegato come è stato realizzato un plugin per QGis per il calcolo del terreno eroso avendo dei dati geomorfici legati al territorio scelto. Questi dati comprendono la conformazione del terreno, le precipitazioni registrate nell'area, il tipo di copertura della vegetazione e il numero di anni per cui questa previsione viene eseguita.

Questo plugin è stato scritto in Python e implementa il plugin teorizzato nella tesi \cite{tesi:ambientale}
% !TEX encoding = UTF-8
% !TEX TS-program = pdflatex
% !TEX root = ../Tesi.tex
% !TEX spellcheck = it-IT

%************************************************
\chapter{Descrizione approfondita dell'argomento}
\label{cap:descrizione}
%************************************************
\section{Problema da risolvere} Lo scopo di questa tesi è quello di calcolare in
modo corretto l'erosione del territorio nell'area geografica scielta.
Questo è stato fatto utilizzando un modello RUSLE basato sull'equazione\footnote{Il
modello proposto è quello indicato in \cite[p.~37]{tesi:ambientale}}

\begin{equation}
A = R * K * LS * C * P
\end{equation}

in cui
\begin{itemize}
\item \textbf{A} corrispondente alla perdita di suolo annua
\item \textbf{R} erosività delle precipitazioni
\item \textbf{K} erodibilità del suolo
\item \textbf{LS} rapporto lunghezza pendenza
\item \textbf{C} fattore di copertura del suolo
\item \textbf{P} misure di prevenzione dell'erosione
\end{itemize}

Questo modello è stato quindi applicato utilizzando i dati in formato di layer raster attraverso l'applicativo QGis.
Questo viene fatto in quanto lo standard \textit{de facto} per questa tipologia di dato è il layer raster.


In particolare questo modello viene utilizzato per calcolare anno per anno l'erosione del territorio interessato avendo variazioni di dati in base all'anno o assumendo come invarianti negli anni i dati inseriti in imput.

\section{QGis}

QGis si definisce come "Un Sistema di Informazione Geografica Libero e Open Source"\cite{site:qgis} ovvero è un software che gestisce, elabora e visualizza dati geomorfici e georeferenziati, completamente gratuito e modificabile in ogni sua parte(è limitato dalla licenza utilizzata nello sviluppo del software \cite{site:cc3}).

In particolare sono state utilizzate le componenti di gestione input e output dei raster, il motore di calcolo e il sistema di rendering video. In oltre il programma supporta anche un sistema di plugin e scripting che aumenta le funzionalità del programma utilizzando codice esterno sviluppato dalla comunnity.

Altro dettaglio molto utile di QGis è il fatto che un programma multi piattaforma. Questo vuol dire che il programma è stato distribuito per più sistemi operativi\footnote{In questo caso per Android, Mac OS, Windows e Linux} e obbliga gli sviluppatori a scrivere plugin e script che siano anch'essi multipiattaforma. Questo permette l'utilizzo dei plugin e degli script per la creazione di funzioni automatizzate utilizzando le componenti già presenti in QGis o aggiungerne di nuove.

In oltre, QGis supporta la quasi totalità dei formati open source utilizzati per la codifica per dati geomorfici e georeferenziati in modo nativo mentre i formati di file che non supporta sono utilizzabili attraverso plugin gratuiti disponibili online.

\subsection{Descrizione dei tipi di dati supportati}

I formati di dati possono essere divisi in tre macrocategorie in base a come vengono forniti:
\begin{itemize}
	\item Vettoriali
	\item Raster
	\item DBRMS con estensione spaziale
\end{itemize}

Questi tre formati permettono solo alcune operazioni possibili legate alla composizione dei dati nel formato di input.
Quindi, per alcune operazioni, QGis converte i dati nell formato appropiato in modo che sia sempre possibile fare tutte le operazioni disponibili, anche se queste puo' portare ad approssimazioni dei dati nella conversione.

\subsection{La struttura di QGis}
QGis è un programma avanzato con molte funzionalità che possono essere suddivise in alcune macro-categorie che sono indipendenti dal tipo di dato o file utilizzato.
Le funzionalità standard di QGis sono gestite attraverso dei moduli che costituiscono l'istallazione di base del programma e le funzionalità aggiuntive vengono implementate attraverso plugin dipendenti da questi moduli di base per poter funzionare correttamente.
Solitamente i moduli sono scritti in C o C++ mentre i plugin e gli script sono in Python.

\subsubsection{Rendering grafico dei dati}
QGis ha la capacità di realizzare svariate visualizzazioni grafiche dei dati inseriti o calcolati.
Questo permette la realizzazioni di immagini ad alta definizione dei dati inseriti e ne permetta anche l'esportazione nei formati più comuni di immagini.

\subsubsection{Realizzazione mappe}
In presenza di dati georeferenziati QGis permette di sovrapporre la rappresentazione grafica dei dati alla mappa corrispondente alle coordinate georeferenziali.
Questo permette di creare mappe fisiche contenenti la visualizzazione di aree di erosione attraverso aree colorate o altri effetti applicabili alla mappa sottostante.

\subsubsection{Creazione, elaborazione e conversione dati}
QGis presenta due moduli per l'elaborazioni di dati:
\begin{itemize}
	\item \textbf{Class} modulo standard di QGis per i calcoli aritmetici elementari e alcuni calcoli statistici di base
	\item \textbf{Grass} modulo aggiuntivo di QGis per i calcoli avanzati e regressioni. Contiene anche modelli previsionali e probabilistici.
\end{itemize}

\subsubsection{Analisi dati}
Quando si lavora con un insieme di dati non è detto che questi siano tutti corretti o non abbiano un livello di approssimazione.
All'interno di QGis sono disponibili, di base, un insieme di funzionalità atte a calcolare la bontà delle approssimazioni o dei dati stessi presi in esame dal programma.
Questo viene fatto di base con algoritmi generici per la maggior parte dei formati ma se si usa solo un tipo specifico o si vuole una approssimazione o un modello preciso è possibile ottenere il risultato desiderato attraverso plugin di terze parti.

\subsection{Sistema dei plugin}
Per aumentare lel funzionalità di QGis e scriptare alcune operazioni sono stati creati i plugin.
Sono dei programmi dipendenti da QGis che implementano funzionalità non presenti nel programma di base o estendono il supporto delle funzioni esistenti o i tipi di file supportati.

\subsubsection{Plugin per formati di file}
Vengono utilizzati per elaborare dati grezzi altrimenti difficilmente manipolabili. 
Un esempio possono essere i dati grezzi ottenuti da degli strumenti di misurazione con dati in output non standard.
Normalmente questi file sono leggibili solo col programma abilitato alle modifiche ma, con l'utilizzo del plugin corretto, è possibile leggerlo e elaborarlo direttamente in QGis.

\subsubsection{Pluing che alterano o aumentano le funzionalità}
La reale forza di QGis. Permettono di ampliare le funzioni del programma semplificando o aumentando le funzionalità pre esisteni. Alcuni esempi sono i plugin della famiglia GRASS che aggiungono operazioni matematiche o gli strumenti di creazione timelapse partendo da una serie di ruster temporizzati.s
% !TEX encoding = UTF-8
% !TEX TS-program = pdflatex
% !TEX root = ../Tesi.tex
% !TEX spellcheck = it-IT

%************************************************
\chapter{Realizzazione}
\label{cap:realizzazione}
%************************************************

\lipsum[1]

\section{Librerie scelte}
\lipsum[2]

\subsection{Qt4}
\subsection{Grass}
\subsection{Osgeo}

\section{Interfaccia grafica}
\lipsum[2]
\subsection{Scelte implementative}

\section{Gestione input}
\lipsum[3]
\subsection{Scelte implementative}

\section{Gestione dati}
\lipsum[4]
\subsection{Scelte implementative}

\section{Gestione output}
\lipsum[5]
\subsection{Scelte implementative}

\appendix
% !TEX encoding = UTF-8
% !TEX TS-program = pdflatex
% !TEX root = ../Tesi.tex
% !TEX spellcheck = it-IT

%************************************************

\chapter{Estendere le funzionalità del plugin}
\label{cap:appendice}
%************************************************
Questa appendice spiega come è strutturato a livello di codice il plugin. In particolare si vuole spiegare dove bisogna mettere mano al codice per modificarlo ed aggiungere funzionalità nuove.

\section{Struttura del plugin}
Il plugin, per poter funzionare, deve seguire la struttura standard indicata sulle api ufficiali \cite{site:qgisapi}:
\begin{itemize}
	\item \textbf{File di configurazione}: necessari per il corretto funzionamento del plugin, contengono le informazioni per far funzionare correttamente il plugin, i codici delle versioni di QGis supportate e gli indirizzi delle risorse grafiche. 
	\item \textbf{Risorse grafiche}: file che contengono il codice che, compilato, produce il codice che disegna le interfacce grafiche del plugin. Questi file sono strati costruiti attraverso la libreria grafica QT e vengono poi compilati in python.
	\item \textbf{Codice}: il codice che viene effettivamente eseguito dal plugin. Diviso in più file per facilità di manutenzione, deve contenere un file \_init\_.py da cui si avvia interamente il plugin.
\end{itemize}

\subsection{Elementi di base}
RUSLECalculator segue alla lettere la stuttura indicata da \cite{site:qgisapi}. Questo comporta la presenza di alcuni file che segnalano a QGis la presenza del plugin e permettano a QGis di caricarlo correttamente.
In particolare è necessaria la presenza di un file \_init\_.py in cui viene lanciato il plugin e un file metadata.txt in cui ci sono i dati richiesti sul plugin. Se questi due file sono assenti o configurati in modo sbagliato o ci sono errori all'interno QGis si rifiuterà di caricarli e, di conseguenza, apparirà nell'elenco dei plugin difettosi (sistema fatto affinché lo sviluppatore possa avere riscontro con QGis su quale sia l'errore incontrato).
In oltre, la corretta e completa compilazione del file metadata.txt permette l'invio del plugin ai server ufficiali di QGis per l'inserimento del plugin all'interno del catalogo ufficiale, che viene visualizzato all'interno del menu Plugin quando si vede quali sono i plugin attivi, disattivati, installati e non installati in QGis.

\subsection{metadata.txt}
File contenente i dati sul plugin. Tutte le informazioni che QGis carica attraverso il catalogo plugin provengono da questo file. 
Questi dati vengono divisi in due macrocategorie:
\begin{itemize}
	\item \textbf{Mandatory items}: Elementi senza i quali QGis non riconosce il plugin. Senza di questi elementi non è nemmeno possibile caricare il plugin nel catalogo online
		\subitem \textbf{name} Nome del plugin che appare nel catalogo e che viene visualizzato nel menu dei plugin
		\subitem \textbf{qgisMinimumVersion} Versione minima richiesta di QGis per far funzionare il plugin. Utile per vedere quanto è retrocompatibile il plugin
		\subitem \textbf{description} Breve descrizione del plugin in modo che sia facilmente comprensibile la sua funzione
		\subitem \textbf{version} Numero di versione del plugin. Essenziale in quanto permette a QGis di capire se è l'ultima versione o c'è una versione successiva che va installata
		\subitem \textbf{author} Nome dell'autore che ha realizzato il plugin
		\subitem \textbf{email} Email di riferimento per il plugin
	\item \textbf{Recommended} Elementi non obbligatori che aggiungono informazioni extra sul materiale. L'elenco che segue sono quelli utilizzati per il plugin
		\subitem \textbf{tags} Elenco di parole chiave separate da virgola che descrivono il plugin. Servono per poter apparire nei risultati del cerca del catalogo dei plugin
		\subitem \textbf{homepage} Pagina del plugin, in cui trovare ulteriori informazioni su di esso
		\subitem \textbf{tracker} Pagina web dove si segnalano tutti i problemi legati al funzionamento del plugin. Solitamente è un servizio che viene offerto dai siti che hostano il repository del plugin
		\subitem \textbf{repository} Link al repository dove è hostato il sorgente del plugin. Il link è alla pagina del progetto corrispondente
		\subitem \textbf{category} Stringa che identifica la tipologia del plugin sviluppato. Utile per essere correttamente assegnato alle categorie del catalogo dei plugin
		\subitem \textbf{icon} Definisce il path relativo dalla cartella del progetto alla immagine utilizzata come icona dal plugin
		\subitem \textbf{experimental} Indica se la versione del plugin è stabile o una versione sperimentale
	\item \textbf{deprecated} Voce per segnalare che l'intero plugin è deprecato. Fare attenzione perchè se segnato come True viene applicato a tutte le versioni del plugin
\end{itemize}

\subsection{GdalTools\_utils.py}
Script dal pacchetto GdalTools. Permette di aprire una finestra di salvataggio del file. Utilizzandola siamo obbligati a utilizzare la licenza GNU GPU v2 o superiore. 
E' stato scelto di utilizzare questo script all'interno del progetto in modo da evitare dipendenze dal pacchetto GdalTools che risulta essere incompatibile con alcune installazioni di python e alcuni plugin. Per questo motivo, seguendo le indicazioni della licenza e condizioni d'uso di GdalTools, è stato copiato in modo integrale il file \textit{GdalTools\_utils.py} contenente la funzione necessaria per la visualizzazione della finestre e dell'elenco dei formati disponibili.

% \appendix
% % !TEX encoding = UTF-8
% !TEX TS-program = pdflatex
% !TEX root = ../Tesi.tex
% !TEX spellcheck = it-IT

%************************************************

\chapter{Estendere le funzionalità del plugin}
\label{cap:appendice}
%************************************************
Questa appendice spiega come è strutturato a livello di codice il plugin. In particolare si vuole spiegare dove bisogna mettere mano al codice per modificarlo ed aggiungere funzionalità nuove.

\section{Struttura del plugin}
Il plugin, per poter funzionare, deve seguire la struttura standard indicata sulle api ufficiali \cite{site:qgisapi}:
\begin{itemize}
	\item \textbf{File di configurazione}: necessari per il corretto funzionamento del plugin, contengono le informazioni per far funzionare correttamente il plugin, i codici delle versioni di QGis supportate e gli indirizzi delle risorse grafiche. 
	\item \textbf{Risorse grafiche}: file che contengono il codice che, compilato, produce il codice che disegna le interfacce grafiche del plugin. Questi file sono strati costruiti attraverso la libreria grafica QT e vengono poi compilati in python.
	\item \textbf{Codice}: il codice che viene effettivamente eseguito dal plugin. Diviso in più file per facilità di manutenzione, deve contenere un file \_init\_.py da cui si avvia interamente il plugin.
\end{itemize}

\subsection{Elementi di base}
RUSLECalculator segue alla lettere la stuttura indicata da \cite{site:qgisapi}. Questo comporta la presenza di alcuni file che segnalano a QGis la presenza del plugin e permettano a QGis di caricarlo correttamente.
In particolare è necessaria la presenza di un file \_init\_.py in cui viene lanciato il plugin e un file metadata.txt in cui ci sono i dati richiesti sul plugin. Se questi due file sono assenti o configurati in modo sbagliato o ci sono errori all'interno QGis si rifiuterà di caricarli e, di conseguenza, apparirà nell'elenco dei plugin difettosi (sistema fatto affinché lo sviluppatore possa avere riscontro con QGis su quale sia l'errore incontrato).
In oltre, la corretta e completa compilazione del file metadata.txt permette l'invio del plugin ai server ufficiali di QGis per l'inserimento del plugin all'interno del catalogo ufficiale, che viene visualizzato all'interno del menu Plugin quando si vede quali sono i plugin attivi, disattivati, installati e non installati in QGis.

\subsection{metadata.txt}
File contenente i dati sul plugin. Tutte le informazioni che QGis carica attraverso il catalogo plugin provengono da questo file. 
Questi dati vengono divisi in due macrocategorie:
\begin{itemize}
	\item \textbf{Mandatory items}: Elementi senza i quali QGis non riconosce il plugin. Senza di questi elementi non è nemmeno possibile caricare il plugin nel catalogo online
		\subitem \textbf{name} Nome del plugin che appare nel catalogo e che viene visualizzato nel menu dei plugin
		\subitem \textbf{qgisMinimumVersion} Versione minima richiesta di QGis per far funzionare il plugin. Utile per vedere quanto è retrocompatibile il plugin
		\subitem \textbf{description} Breve descrizione del plugin in modo che sia facilmente comprensibile la sua funzione
		\subitem \textbf{version} Numero di versione del plugin. Essenziale in quanto permette a QGis di capire se è l'ultima versione o c'è una versione successiva che va installata
		\subitem \textbf{author} Nome dell'autore che ha realizzato il plugin
		\subitem \textbf{email} Email di riferimento per il plugin
	\item \textbf{Recommended} Elementi non obbligatori che aggiungono informazioni extra sul materiale. L'elenco che segue sono quelli utilizzati per il plugin
		\subitem \textbf{tags} Elenco di parole chiave separate da virgola che descrivono il plugin. Servono per poter apparire nei risultati del cerca del catalogo dei plugin
		\subitem \textbf{homepage} Pagina del plugin, in cui trovare ulteriori informazioni su di esso
		\subitem \textbf{tracker} Pagina web dove si segnalano tutti i problemi legati al funzionamento del plugin. Solitamente è un servizio che viene offerto dai siti che hostano il repository del plugin
		\subitem \textbf{repository} Link al repository dove è hostato il sorgente del plugin. Il link è alla pagina del progetto corrispondente
		\subitem \textbf{category} Stringa che identifica la tipologia del plugin sviluppato. Utile per essere correttamente assegnato alle categorie del catalogo dei plugin
		\subitem \textbf{icon} Definisce il path relativo dalla cartella del progetto alla immagine utilizzata come icona dal plugin
		\subitem \textbf{experimental} Indica se la versione del plugin è stabile o una versione sperimentale
	\item \textbf{deprecated} Voce per segnalare che l'intero plugin è deprecato. Fare attenzione perchè se segnato come True viene applicato a tutte le versioni del plugin
\end{itemize}

\subsection{GdalTools\_utils.py}
Script dal pacchetto GdalTools. Permette di aprire una finestra di salvataggio del file. Utilizzandola siamo obbligati a utilizzare la licenza GNU GPU v2 o superiore. 
E' stato scelto di utilizzare questo script all'interno del progetto in modo da evitare dipendenze dal pacchetto GdalTools che risulta essere incompatibile con alcune installazioni di python e alcuni plugin. Per questo motivo, seguendo le indicazioni della licenza e condizioni d'uso di GdalTools, è stato copiato in modo integrale il file \textit{GdalTools\_utils.py} contenente la funzione necessaria per la visualizzazione della finestre e dell'elenco dei formati disponibili.

% *****************************************************************
% Materiale finale
%******************************************************************
\backmatter
\input{MaterialeInizialeFinale/Bibliografia}
% \input{MaterialeInizialeFinale/Dichiarazione}
\end{document}
