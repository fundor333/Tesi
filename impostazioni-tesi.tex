%*********************************************************************************
% Comandi persaonali
%*******************************************************
\newcommand{\myName}{Lorenzo Pantieri}                       % autore
\newcommand{\myTitle}{La prova ontologica dell'esistenza di Dio} % titolo
\newcommand{\myDegree}{Tesi di laurea}                       % tipo di tesi
\newcommand{\myUni}{Universit\`a degli Studi del Magdeburgo} % universit\`a
\newcommand{\myFaculty}{Facolt\`a di Lettere e Filosofia}    % facolt\`a
\newcommand{\myDepartment}{Dipartimento di Teologia}         % dipartimento
\newcommand{\myProf}{Chiar.mo Prof.~S.~Anselmo d'Aosta}      % relatore
%\newcommand{\myOtherProf}{Dott.~Immanuel Kant}              % eventuale correlatore
\newcommand{\myLocation}{Magdeburgo}                         % dove
\newcommand{\myTime}{Dicembre 2011}                          % quando



%*********************************************************************************
% Impostazioni di amsmath, amssymb, amsthm
%*********************************************************************************

% comandi per gli insiemi numerici (serve il pacchetto amssymb)
\newcommand{\numberset}{\mathbb}
\newcommand{\N}{\numberset{N}}
\newcommand{\R}{\numberset{R}}

% un ambiente per i sistemi
\newenvironment{sistema}%
  {\left\lbrace\begin{array}{@{}l@{}}}%
  {\end{array}\right.}

% definizioni (serve il pacchetto amsthm)
\theoremstyle{definition}
\newtheorem{definizione}{Definizione}

% teoremi, leggi e decreti (serve il pacchetto amsthm)
\theoremstyle{plain}
\newtheorem{teorema}{Teorema}
\newtheorem{legge}{Legge}
\newtheorem{dimostrazione}{Dimostrazione}
\newtheorem{decreto}[legge]{Decreto}
\newtheorem{murphy}{Murphy}[section]



%*********************************************************************************
% Impostazioni di biblatex
%*********************************************************************************
\defbibheading{bibliography}{%
\cleardoublepage
\phantomsection
\addcontentsline{toc}{chapter}{\bibname}
\chapter*{\bibname\markboth{\bibname}
{\bibname}}}


%*********************************************************************************
% Impostazioni di listings
%*********************************************************************************
\lstset{language=[LaTeX]Tex,%C++,
    keywordstyle=\color{RoyalBlue},%\bfseries,
    basicstyle=\small\ttfamily,
    %identifierstyle=\color{NavyBlue},
    commentstyle=\color{Green}\ttfamily,
    stringstyle=\rmfamily,
    numbers=none,%left,%
    numberstyle=\scriptsize,%\tiny
    stepnumber=5,
    numbersep=8pt,
    showstringspaces=false,
    breaklines=true,
    frameround=ftff,
    frame=single
}



%*********************************************************************************
% Impostazioni di hyperref
%*********************************************************************************
\hypersetup{%
    hyperfootnotes=false,pdfpagelabels,
    %draft,	% = elimina tutti i link (utile per stampe in bianco e nero)
    colorlinks=true, linktocpage=true, pdfstartpage=1, pdfstartview=FitV,%
    % decommenta la riga seguente per avere link in nero (per esempio per la stampa in bianco e nero)
    %colorlinks=false, linktocpage=false, pdfborder={0 0 0}, pdfstartpage=1, pdfstartview=FitV,%
    breaklinks=true, pdfpagemode=UseNone, pageanchor=true, pdfpagemode=UseOutlines,%
    plainpages=false, bookmarksnumbered, bookmarksopen=true, bookmarksopenlevel=1,%
    hypertexnames=true, pdfhighlight=/O,%nesting=true,%frenchlinks,%
    urlcolor=webbrown, linkcolor=RoyalBlue, citecolor=webgreen, %pagecolor=RoyalBlue,%
    %urlcolor=Black, linkcolor=Black, citecolor=Black, %pagecolor=Black,%
    pdftitle={\myTitle},%
    pdfauthor={\textcopyright\ \myName, \myUni, \myFaculty},%
    pdfsubject={},%
    pdfkeywords={},%
    pdfcreator={pdfLaTeX},%
    pdfproducer={LaTeX with hyperref and ClassicThesis}%
}


%*********************************************************************************
% Impostazioni di graphicx
%*********************************************************************************
\graphicspath{{Immagini/}} % cartella dove sono riposte le immagini

%*********************************************************************************
% Impostazioni di xcolor
%*********************************************************************************
\definecolor{webgreen}{rgb}{0,.5,0}
\definecolor{webbrown}{rgb}{.6,0,0}

%*********************************************************************************
% Impostazioni di caption
%*********************************************************************************
\captionsetup{tableposition=top,figureposition=bottom,font=small,format=hang,labelfont=bf}

%*********************************************************************************
% Altro
%*********************************************************************************

% [...] ;-)
\newcommand{\omissis}{[\dots\negthinspace]}

% eccezioni all'algoritmo di sillabazione
\hyphenation{Fortran ma-cro-istru-zio-ne nitro-idrossil-amminico}
